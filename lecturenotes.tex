%% LaTeX Paper Template, Flip Tanedo (flip.tanedo@ucr.edu)
%% GitHub: https://github.com/fliptanedo/paper-template-2022

% \documentclass[11pt]{article} %% Not for Lecture Notes
\documentclass[12pt, oneside]{report}    %% Has chapters


%%%%%%%%%%%%%%%%%%%%%%%%%%%%%%%%%%%%%%%%%%%%%%%%%%%%%%%%%%%

%% BibLaTeX does not want the cite package
%% from: https://tex.stackexchange.com/a/39418/8094
\makeatletter
\newcommand{\disablepackage}[2]{%
  \disable@package@load{#1}{#2}%
}
\newcommand{\reenablepackage}[1]{%
  \reenable@package@load{#1}%
}
\makeatother
% 
\disablepackage{cite}{}

%%%%%%%%%%%%%%%%%%%%%%%%%%%%%%%%%%%%%%%%%%%%%%%%%%%%%%%%%%%


%!TEX root = paper.tex
%% FLIP’S PREAMBLE; 
%% Use FlipAdditionalHeader for project-specific packages & macros
%% Leave this as general as possible

%%%%%%%%%%%%%%%%%%%%%%%%%%
%%%  COMMON PACKAGES  %%%%
%%%%%%%%%%%%%%%%%%%%%%%%%%

\usepackage{amsmath}            % AMS Macros
\usepackage{amssymb}            %
\usepackage{amsfonts}           %
\usepackage{amsthm}             % 

\usepackage{graphicx}           % includegraphics
\usepackage[utf8]{inputenc}     % inspire bibs
\usepackage{aas_macros}				  % ADS bibs
\usepackage{bm}                 % \boldsymbol
\usepackage{microtype}          % improved typogarphy
\usepackage{etoolbox}           % LaTeX primitives

%%%%%%%%%%%%%%%%%%%%%%%%%%%
%%%  UNUSUAL PACKAGES  %%%%
%%%%%%%%%%%%%%%%%%%%%%%%%%%

%% MATH AND PHYSICS SYMBOLS
%% ------------------------
\usepackage{slashed}				% \slashed{k}
\usepackage{mathrsfs}				% Weinberg-esque letters
\usepackage{bbm}					  % \mathbbm{1} conflict: XeLaTeX 
\usepackage{cancel}					% cross out
\usepackage[normalem]{ulem} % for \sout
\usepackage{youngtab}	    	% Young Tableaux
\usepackage{mleftright}     % \mleft, \mright; bracket size/spacing

%% CONTENT FORMAT AND DESIGN
%% -------------------------
\usepackage[dvipsnames]{xcolor}
\usepackage[hang,flushmargin]{footmisc} % no footnote indent

\usepackage{fancyhdr}		% preprint number
\usepackage{lipsum}			% block of text 
% \usepackage{tcolorbox}  % replace framed and mdframed
\usepackage[most]{tcolorbox} % `most' needed for listings
\usepackage{subcaption}	% subfigures
\usepackage{cite}			  % group cites
\usepackage{wrapfig}    

%% TABLES IN LaTeX
%% ---------------
\usepackage{booktabs}		% professional tables
\usepackage{nicefrac}		% fractions in tables,
\usepackage{multirow}		% multirow elements in a table
\usepackage{arydshln}		% dashed lines in arrays

%% ARRAY STRETCH: vertical spacing between rows
% \renewcommand{\arraystretch}{1.5} %% put this in main text

%% Other Packages and Notes
%% ------------------------
\usepackage[font=small]{caption} 	% caption font is small
\usepackage{float}         			  % for strict placement e.g. [H]
\usepackage{lineno}               % Line numbers (put \linenumbers in main text)
\usepackage{ccicons}              % Creative Commons License Icons

%%%%%%%%%%%%%%%%%%%%%%%%%%%%%%
%%%  DOCUMENT PROPERTIES  %%%%
%%%%%%%%%%%%%%%%%%%%%%%%%%%%%%

\usepackage[margin=2.5cm]{geometry} % margins
\usepackage{changepage}             % overwrite geometry (e.g. lecturenotes)
\numberwithin{equation}{section}    % set equation numbering
\usepackage{marginnote}             % for \marginnote{comment}
% \usepackage{mparhack}               % fix for \marginnote
% \usepackage{marginfix}              % fix for \marginnote
% \usepackage{adjustbox}              % to rescale elements

%% References in two columns, smaller
%% http://tex.stackexchange.com/questions/20758/
\usepackage{multicol}
% \usepackage{etoolbox} %% called above
\usepackage{relsize}
\patchcmd{\thebibliography}
  {\list}
  {\begin{multicols}{2}\smaller\list}
  {}
  {}
\appto{\endthebibliography}{\end{multicols}}

% Change list spacing (instead of package paralist)
% from: http://en.wikibooks.org/wiki/LaTeX/List_Structures#Line_spacing
% alternative: enumitem package
\let\oldenumerate\enumerate
\renewcommand{\enumerate}{
  \oldenumerate
  \setlength{\itemsep}{4pt}
  \setlength{\parskip}{0pt}
  \setlength{\parsep}{0pt}
}

\let\olditemize\itemize
\renewcommand{\itemize}{
  \olditemize
  \setlength{\itemsep}{4pt}
  \setlength{\parskip}{0pt}
  \setlength{\parsep}{0pt}
}



%%%%%%%%%%%%%%%%%%%%%
%%%  TITLE DATA  %%%%
%%%%%%%%%%%%%%%%%%%%%

%% COMMANDS FOR TOP-MATTER
%% -----------------------
\newcommand{\email}[1]{\href{mailto:#1}{#1}}
% \newenvironment{institutions}[1][2em]{\begin{list}{}{\setlength\leftmargin{#1}\setlength\rightmargin{#1}}\item[]}{\end{list}}
%% ... old

%% PREPRINT NUMBER USING fancyhdr
%% Don't forget to set \thispagestyle{firststyle}
%% ----------------------------------------------
\renewcommand{\headrulewidth}{0pt}  % no separator
\setlength{\headheight}{15pt}     % min to avoid fancyhdr warning
\fancypagestyle{firststyle}{
  \rhead{\footnotesize%
  \texttt{\FlipTR}%
  }}

%% TOC overwrites fancyhdr, here's a fix
%% http://tex.stackexchange.com/questions/167828/
\usepackage{etoc}
\renewcommand{\etocaftertitlehook}{\pagestyle{plain}}
\renewcommand{\etocaftertochook}{\thispagestyle{firststyle}}



%%%%%%%%%%%%%%%%%%%%%%%%%%%
%%%  (RE)NEW COMMANDS  %%%%
%%%%%%%%%%%%%%%%%%%%%%%%%%%

%% COMMANDS FOR LATEXDIFF
%% ----------------------
%% see http://bit.ly/1M74uwc
\providecommand{\DIFadd}[1]{{\protect\color{blue}#1}} %DIF PREAMBLE
\providecommand{\DIFdel}[1]{{\protect\color{red}\protect\scriptsize{#1}}}

%% REMARK: use latexdiff option --allow-spaces
%% for \frac, ref: http://bit.ly/1iFlujR
%% Errors with environments? https://tex.stackexchange.com/q/73224

%% USAGE: latexdiff draft.tex revision.tex > diff.tex


			%% \usepackages, formatting
%!TEX root = paper.tex
%% FLIP’S MACROS 
%% USES: FlipPreamble.tex

%% FOR `NOT SHOUTING' CAPS (e.g. acronyms)
%% ---------------------------------------
\usepackage{scalefnt} 
% \newcommand\acro[1]{{\footnotesize #1}}           % acronyms in footnote size
\newcommand\acro[1]{{\small{#1}}} 


%% COMMANDS FOR TEMPORARY COMMENTS
%% -------------------------------
\newcommand{\comment}[2]{\textcolor{red}{[\textbf{#1} #2]}}
\newcommand{\flip}[1]{{
  \color{green!50!black}
  \footnotesize
  [\textbf{\textsf{Flip}}: \textsf{#1}]
  }}

\newcommand{\new}[1]{{ 
    \color{green!50!black}\footnotesize
    [\textbf{\textsf{New}}: {#1}]}}

%% FRAMES (mdframed package)
\global\mdfdefinestyle{flipbox}{%
  linecolor=green!50!black,linewidth=1pt,%
  leftmargin=0cm, rightmargin=0cm
  } 


\newenvironment{flipcomment}
  {
    \begin{mdframed}[style=flipbox] 
    \color{green!50!black}
    \sffamily
    \footnotesize
    \textbf{\textsf{Flip}}:
  }{
    \end{mdframed}
  }

%% Because mdframed causes a bunch of warnings
%% https://tex.stackexchange.com/questions/64331/disable-warning-from-mdframed 
\usepackage{silence}
\WarningFilter{mdframed}{You got a bad break}
\WarningFilter{mdframed}{correct box splittet fails}



%% REMOVE Environment
%% https://tex.stackexchange.com/a/488582/8094
%% Creates a nolabel environment that strips all labels
%% This is useful to avoid multiple label definitions
%% When marking old versions for deletion
\usepackage{xparse}
\ExplSyntaxOn
\NewDocumentEnvironment{nolabel}{}{
  \cs_set_eq:NN \label \use_none:n
  \cs_set_eq:cN { ltx@label} \use_none:n
}{}
\ExplSyntaxOff 

\newcommand{\remove}[1]{{
  \begin{nolabel} 
    \color{red!50!black}\footnotesize
    [\textbf{\textsf{Removed}}: {#1}]
    \end{nolabel}
    }}



%% COMMON PHYSICS MACROS
%% ---------------------
\renewcommand{\tilde}{\widetilde}                 % tilde over characters
\renewcommand{\text}{\textnormal}	                % text in equations 
\renewcommand{\vec}[1]{\mathbf{#1}}               % vectors are boldface

%% Differential and differential/2pi
% \newcommand{\dbar}{d\mkern-6mu\mathchar'26}     % for d/2pi
\newcommand{\dbar}{d\mkern-6mu\mathchar'26\hspace{-.1em}}    

%% Best practice: Roman differential
\newcommand{\D}[1]{\ensuremath{\operatorname{d}\!{#1}}}
\newcommand{\DD}[2]{\ensuremath{\operatorname{d}^{#1}\!{#2}}}
\newcommand{\Dbar}[1]{\operatorname{d}\mkern-10mu\mathchar'26\mkern-2mu{#1}} 

\newcommand{\ket}[1]{\left|#1\right\rangle}       % <#1|
\newcommand{\bra}[1]{\left\langle#1\right|}       % |#1>

%% Best practice: base of natural log is Roman
\newcommand{\e}{\operatorname{e}}  

%% Best practice: imaginary number is Roman too!?
\newcommand{\I}{\operatorname{i}\mkern-2mu}  


%% Matrices
%% --------
\newcommand{\aij}[2]{^{#1}_{\phantom{#1}#2}}
\newcommand{\mat}[3]{#1\aij{#2}{#3}}
\newcommand{\pp}{\phantom{+}}                     % phantom + for spacing
\newcommand{\inv}{^{-1}}
\newcommand{\one}{\mathbbm{1}}
\newcommand{\Tr}{\text{Tr}\,}
\newcommand{\RR}{\mathbbm{R}}
\newcommand{\CC}{\mathbbm{C}}
\newcommand*{\trans}{\mkern-1.5mu\mathsf{T}}      % transpose


% Make my life easer
% ------------------
\newcommand{\la}{\langle}
\newcommand{\ra}{\rangle}
\newcommand*{\smallslot}{\,\underline{\makebox[0.80em]{\ensuremath{}}}\,}


\usepackage{scalerel} % https://tex.stackexchange.com/a/523873/8094
\newcommand*{\paral}{{\stretchrel*{\parallel}{\perp}}}


\usepackage{pifont}
  \newcommand{\cmark}{\ding{51}}%
  \newcommand{\xmark}{\ding{55}}%

% For := with the dots and lines aligned, same size
% h/t tex.stackexchange.com/a/4881/8094
\newcommand*{\defeq}{\mathrel{\vcenter{\baselineskip0.5ex \lineskiplimit0pt
                     \hbox{\scriptsize.}\hbox{\scriptsize.}}}%
                     =}              %% Flip's standard macros
%!TEX root = paper.tex
%% FLIP’S MACROS FOR COMMENTS
%% USES: FlipPreamble.tex
%%
%% These macros are for communicating between collaborators
%% during the editing process.

%% USAGE SUMMARY: EXAMPLES
%% -----------------------
%% \comment{Check}{Is this equation correcT?}
%% \comment{Flip}{I think it is correct.}
%%
%% \begin{flipcomment}
%% This is a more involved comment, perhaps with equations
%% \end{flipcomment}
%%
%% \comment{Flip}{Can we discuss adding this:}
%% \new{This chunk of text is new compared to the last version}
%%
%% \comment{Flip}{Can we discuss removing this?}
%% \remove{Previous version of the text that I propose removing.} 




%% INLINE COMMENTS
%% ---------------
\newcommand{\comment}[2]{\textcolor{red}{[\textbf{#1}: #2]}}

%% SHORT COMMENT: copy this to make your own inline comment
\newcommand{\flip}[1]{{
  \color{green!50!black}
  \footnotesize
  [\textbf{\textsf{Flip}}: \textsf{#1}]
  }}


%% IN-BOX COMMENMTS (for longer comments)
%% --------------------------------------

%% Define frame (mdframed package)
\global\mdfdefinestyle{flipbox}{%
  linecolor=green!50!black,linewidth=1pt,%
  leftmargin=0cm, rightmargin=0cm
  } 

%% LONG COMMENT: copy this to make your own boxed comment
\newenvironment{flipcomment}
  {
    \begin{mdframed}[style=flipbox] 
    \color{green!50!black}
    \sffamily
    \footnotesize
    \textbf{\textsf{Flip}}:
  }{
    \end{mdframed}
  }

%% Because mdframed causes a bunch of warnings
%% https://tex.stackexchange.com/questions/64331/disable-warning-from-mdframed 
\usepackage{silence}
\WarningFilter{mdframed}{You got a bad break}
\WarningFilter{mdframed}{correct box splittet fails}



%% ADDING AND REMOVING TEXT
%% ------------------------
%% Analogous to LaTeXdiff-by-hand

\newcommand{\new}[1]{{ 
    \color{green!50!black}\footnotesize
    [\textbf{\textsf{New}}: {#1}]}}


%% REMOVE Environment
%% https://tex.stackexchange.com/a/488582/8094
%% Creates a nolabel environment that strips all labels
%% This is useful to avoid multiple label definitions
%% When marking old versions for deletion
\usepackage{xparse}
\ExplSyntaxOn
\NewDocumentEnvironment{nolabel}{}{
  \cs_set_eq:NN \label \use_none:n
  \cs_set_eq:cN { ltx@label} \use_none:n
}{}
\ExplSyntaxOff 

\newcommand{\remove}[1]{{
  \begin{nolabel} 
    \color{red!50!black}\footnotesize
    [\textbf{\textsf{Removed}}: {#1}]
    \end{nolabel}
    }}
     %% Flip's macros for comments
%!TEX root = paper.tex
%% FLIP’S MACROS FOR COMMENTS
%% USES: FlipPreamble.tex FlipMacros_Comments.tex
%%
%% These macros are for course notes.

%% USAGE SUMMARY: EXAMPLES
%% -----------------------
%% \begin{exercise}
%%   Solve the following differential equation..
%%   \label{ex:solve:ode}
%% \end{exercise} 

\usepackage{appendix}   % for sub-appendices
%                         % see https://tex.stackexchange.com/a/120723/8094
%                         % ... and discussion therein

% \theoremstyle{plain} % default
% \theoremstyle{remark} % upright text, no extra space above or below

% From amsthm documentation and https://tex.stackexchange.com/a/38264/8094
\newtheoremstyle{flip}% <name>
{0pt}% <Space above>
{0pt}% <Space below>
{}% <Body font>
{}% <Indent amount>
{\bfseries}% <Theorem head font>
{.}% <Punctuation after theorem head>
{.5em}% <Space after theorem headi>
{}% <Theorem head spec (can be left empty, meaning `normal')>
\theoremstyle{flip}

\newtheorem{theorem}{Theorem}[section]
\newtheorem{exercise}{Exercise}[section]
\newtheorem{example}{Example}[section]
\newtheorem{bigidea}{Key Idea}[section]
    \newcommand{\bigidearef}{Key~Idea}
    \newcommand{\bigidearefs}{Key~Ideas}

\AtBeginEnvironment{example}{\footnotesize}
\AtBeginEnvironment{exercise}{\footnotesize}


% TCOLORBOX settings
% https://tex.stackexchange.com/a/633497/8094
\tcolorboxenvironment{theorem}{
    enhanced, % Skin Family `enhanced'
    frame hidden,  % no frame
    interior hidden, % no interior color
    breakable, % allows box to flow across pages
    borderline west={2pt}{0pt}{gray}
    }

\tcolorboxenvironment{exercise}{
    enhanced, % Skin Family `enhanced'
    frame hidden,  % no frame
    interior hidden, % no interior color
    breakable, % allows box to flow across pages
    borderline west={2pt}{0pt}{red!50!black}
    }

\tcolorboxenvironment{example}{
    enhanced, % Skin Family `enhanced'
    frame hidden,  % no frame
    interior hidden, % no interior color
    breakable, % allows box to flow across pages
    borderline west={2pt}{0pt}{green!50!black}
    }

\tcolorboxenvironment{bigidea}{
    enhanced, % Skin Family `enhanced'
    frame hidden,  % no frame
    interior hidden, % no interior color
    breakable, % allows box to flow across pages
    borderline west={2pt}{0pt}{blue!50!black}
    }
     %% Flip's macros for course notes
%!TEX root = paper.tex
%% Update the above with the appropriate root

%% Place additional project-specific macros, package calls here
%% These are called before FlipPreambleEnd.tex so that,
%% for example, they are called before hyperref

%% Example: replace YourName with your name
\newcommand{\YourName}[1]{{	
	\color{blue!50!black}\footnotesize
	[\textbf{\textsf{YourName}}: \textsf{#1}]}}

%% ACRONYMS
\newcommand{\DM}{\acro{DM}\xspace}		% nb: I do not like using this


%% COMMANDS FOR TEMPORARY COMMENTS
%% -------------------------------
\newcommand{\comment}[2]{\textcolor{red}{[\textbf{#1} #2]}}
\newcommand{\flip}[1]{{
	\color{green!50!black}
  \footnotesize
  [\textbf{\textsf{Flip}}: \textsf{#1}]
	}}

%% FRAMES (mdframed package)

\global\mdfdefinestyle{flipbox}{%
	linecolor=green!50!black,linewidth=1pt,%
	leftmargin=0cm, rightmargin=0cm
} 

\newenvironment{flipcomment}
	{
		\begin{mdframed}[style=flipbox] 
		\color{green!50!black}
		\sffamily
		\footnotesize
 		\textbf{\textsf{Flip}}:
	}{
		\end{mdframed}
	}
 	%% Modify this for each project
\input{Flip_listings}           %% Styling for code blocks
%!TEX root = paper.tex
%% FLIP’S MACROS 
%% USES: FlipPreamble.tex

% Macros for lecture note typesettingj

       %% Lecture note formatting

\usepackage{sidenotes}  %% no longer maintained
\renewcommand*\thesidenote{\alph{sidenote}}  %% sidenotes indexed letters

% Reset sidenote numbering
\let\oldchapter\chapter
\def\chapter{%
  \setcounter{sidenote}{1}%
  \oldchapter
}


%% Sidenote font and size
%% PART I: https://tex.stackexchange.com/a/536083/8094
%% n.b. a/532251/8094 broke the sidenote floating
    \usepackage{xparse}
    \let\oldmarginpar\marginpar
    \RenewDocumentCommand{\marginpar}{om}{%
      \IfNoValueTF{#1}
        {\oldmarginpar{\mymparsetup #2}}
        {\oldmarginpar[\mymparsetup #1]{\mymparsetup #2}}}

    %% Old answer in a/532251/8094 makes all sidenotes marginnotes
    %% New answer (above) uses marginpar 
    % \renewcommand*{\marginfont}{\sffamily}    % Old
    \newcommand{\mymparsetup}{\scriptsize\sffamily}        % New, using xparse
    % \renewcommand{\sidecaption}{\scriptsize\sffamily}


%% Sans Serif Font Option for Sidenote
\usepackage[thin, scaled=1]{FiraSans} 


% Define lecture note geometry
\geometry{
    paper=letterpaper, 
    hmargin={1cm,7.25cm}, 
    vmargin={2cm,2cm}, 
    marginparsep=.5cm, 
    marginparwidth=5.75cm
}

%% If we need to use the full page
\usepackage{changepage}
\newenvironment{wide}{\begin{adjustwidth}{0cm}{-6.25cm}}{\end{adjustwidth}}
% NOTE: 6.25cm is to match the hmargin difference


% \usepackage{sidecap}
% \sidecaptionvpos{figure}{t}

% \usepackage[font=scriptsize,format=,labelfont={bf,sf},width=10pt]{caption}

\DeclareCaptionStyle{sidecaption}{font={sf,footnotesize}}

% BIBLATEX: for foot citations
\usepackage[style=verbose]{biblatex}
\addbibresource{FlipBib.bib}


\input{FlipPreambleEnd}			%% packages that have to be at the end

% \usepackage[font={small, sf}]{caption}
% For editing
% \linenumbers                  %% print line numbers (lineno package)

\graphicspath{{figures/}}       %% figure folder




% NOTES: to implement
% https://www.overleaf.com/latex/examples/footnote-citations-with-biblatex/sdnfkjhrnpkp
% but biblatex is a pain... no comparable style to utcaps
% https://tex.stackexchange.com/questions/12806/guidelines-for-customizing-biblatex-styles

% \captionsetup[figure*]{font={footnotesize,sf}}
% For figure*  and marginfigure
\DeclareCaptionStyle{widefigure}{font={footnotesize,sf}}

\begin{document}

\newgeometry{margin=2cm}                   % plain geometry for frontmatter
\newcommand{\FlipTR}{UCR-TR-2024-FLIP-00X} % TR#, pdfsync may fail on 1st page
\thispagestyle{firststyle} 	               % TR#; otherwise use \thispagestyle{empty}
\pagenumbering{gobble}                     % no page number on first page 

%%%%%%%%%%%%%%%%%%%%%%%%
%%%  FRONTMATTER    %%%%
%%%%%%%%%%%%%%%%%%%%%%%%


\begin{center}
    {\large \textsc{UC Riverside Physics XXX, Fall 20YY} \par}
    {\huge \textbf{Course Title} \par}\vspace{.5em}
    {\large {Descriptive subtitle} \par}
    \vskip .5cm
\end{center}

\input{FlipAuthors}

\vspace{2em}\noindent
Course description, abstract for lecture notes, etc. \lipsum[5]

\vspace{2em}
\noindent
{Last Compiled: \today}

\noindent % Course notes URL
% \url{https://github.com/fliptanedo/P231-2023-Math-Methods}

%% Front page logos
\vspace*{\fill}
\begin{center}
\includegraphics[height=.1\textwidth]{figures/FlipAmbigram.png}
\hspace{5em}
\includegraphics[height=.1\textwidth]{figures/UCRPnA_banner.png}
\end{center}

\newpage

\small
\setcounter{tocdepth}{2}
\tableofcontents
\normalsize
\clearpage
\restoregeometry        %% Return to lecture note geometry 
\pagenumbering{arabic}  %% Turn on regular page numbers


%%%%%%%%%%%%%%%%%%%%%
%%%  THE CONTENT  %%%
%%%%%%%%%%%%%%%%%%%%%

\chapter{Lecture Note Demonstration} %% if using report class


\section{Vision}

This document is inspired by Edward Tufte by way of the \texttt{tufte-latex} package. The notes have a large margin for side notes and floats (e.g.\ figures).  Visually this means that the column of main text is narrower, which permits a slightly smaller font.\begin{marginfigure}%[th]
    % \centering
    \includegraphics[width=.8\textwidth]{example-image-golden}
    \captionsetup{font={scriptsize,sf}}
    \caption{Example of a margin figure.}
    \label{fig:figure:example:golden}
\end{marginfigure}

% \lipsum[2]

\section{Using the Margin}

We implement marginalia with the \texttt{sidenotes} package. We highlight the main usage here. A standard sidenote uses \verb!\sidenote{...}! and looks like this\sidenote{Test of a sidenote. Let's add some extra text here to demonstrate the following point about non-overlapping notes}. Unlike \verb!marginnote!s, \verb!sidenotes! do not overlap with each other.\sidenote{An example of a sidenote that does not overlap with the previous one.} What more, the sidenotes coexist fine with footnotes.\footnote{Here is a foot note.}.

\paragraph{Figures} You can place entire figure floats in the main text region or in the margin. Fig.~\ref{fig:figure:example:golden} is a good example.

All we did was take \texttt{figure}$\rightarrow$\texttt{marginfigure}. You can do the same thing with \texttt{margintable}. On the other hand, sometimes you want a figure that spans the entire text width. To do this, we simply use the \texttt{figure*} environment. We demonstrate this in Fig.~\ref{fig:subfigure:example:lec}.
\begin{figure*}%[th]
    \centering
    \begin{subfigure}{0.3\linewidth}
    \centering
        \includegraphics[width=\linewidth]{example-image-a}
        \caption{First subfig}
        \label{fig:subfig:1:lec}
    \end{subfigure}\;%
    \begin{subfigure}{0.3\linewidth}
    \centering
        \includegraphics[width=\linewidth]{example-image-a}
        \caption{Second subfig}
        \label{fig:subfig:2:lec}
    \end{subfigure}\;%
    \begin{subfigure}{0.3\linewidth}
    \centering
        \includegraphics[width=\linewidth]{example-image-a}
        \caption{Third subfig}
        \label{fig:subfig:3:lec}
    \end{subfigure}%
    % \captionsetup{font={footnotesize,sf}}
    \caption{Here's how to spread a figure across the entire page, not just the main text width. \lipsum[1]}
    \label{fig:subfigure:example:lec}
\end{figure*}

If we want something in between, we can place the figure in the main text and then place the figure caption in the margin notes. 
\begin{figure}%[th]
    % \centering
    \sidecaption[][-2\baselineskip]{Example of a margin figure. \lipsum[3]}
    \includegraphics[width=\textwidth]{example-image-golden}
    \label{fig:figure:example:golden:sidecap}
\end{figure}


\section{Breaking the Margin}


We define an environment \texttt{wide} that allows text, like equations, to spill into the margins. For example:
\begin{wide}
\begin{align}
f &= \sin\mleft(\frac{x^2}{2}\mright)
\times \arctan t 
\times \log \mleft(\cos \theta\mright)
\times \int_a^b \D{}x \exp\mleft(a^1 + b^2 + x^2\mright)
\times e^{-i\pi} 
\times \Gamma(n) 
\times _{n\!}\text{C}_m
\end{align}
\end{wide}
\begin{wide}
The definition of the margin spillover in \texttt{wide} needs to be matched to the size of the margin defined with the \texttt{geometry} package.
Here's what normal text looks like.  \lipsum[2]
... just don't mix this with sidenotes. 
\end{wide}


% Here's a sentence with some citations.\sidenote[a]{\footnotesize \textsf{And another note. Can I make this text smaller.}}\sidenote[a]{Another}\lipsum[1]

Here's a sentence with some citations.\sidenote{
    \scriptsize\lipsum[10]}

\lipsum[1]\sidenote{Another} \lipsum[1]

% And here's a sentence. \marginnote{\footnotesize \textsf{And another note. Can I make this text smaller.}}
% \marginnote{Here's another one}

% What I'm working on:
% % https://ctan.math.illinois.edu/macros/latex/contrib/sidenotesplus/tests-sidenoteplus.pdf
% % https://damitr.org/2020/03/10/sidenotes-and-label-in-latex/

\paragraph{Example \emph{Lorem Ipsum} text} \lipsum[5]

% %!TEX root = paper.tex
%% Update the above with the appropriate root


\section{Common environments}


\subsection{Figures: floating and wrapped}

\begin{figure}%[th]
    \centering
    \includegraphics[width=0.4\textwidth]{example-image-a}
    \caption{The figure environment shows up often.}
    \label{fig:figure:example}
\end{figure}

Use \textbackslash\texttt{centering} rather than the \texttt{center} environment in figure environments to avoid adding extra vertical space.\footnote{\url{https://tex.stackexchange.com/a/23653/8094}\label{foot:centering}}


\begin{wrapfigure}{l}{0.3\textwidth}
	\includegraphics[width=0.9\linewidth]{example-image-a}
	\caption{via \texttt{wrapfigure}.}
	\label{fig:wrapfig}
\end{wrapfigure}
\lipsum[1]

\subsection{Figures in Equation Environments}
\label{sec:figs}

\begin{align}
	\vcenter{
		\hbox{\includegraphics[width=.1\textwidth]{{example-image-a}}}
		}
	&=
	i g \gamma^\mu \ . 
	\label{eq:vector}
	\\
	\vcenter{
		\hbox{\includegraphics[width=.1\textwidth]{{example-image-a}}}
		}
	&=
	g \gamma^\mu\gamma^5 \ . 
	\label{eq:axial}
	\\
	\vcenter{
		\hbox{\includegraphics[width=.1\textwidth]{{example-image-a}}}
		}
	&=
	ig  \ . 
	\label{eq:scalar}
	\\
	\vcenter{
		\hbox{\includegraphics[width=.1\textwidth]{{example-image-a}}}
		}
	&=
	g \gamma^5 \ . 
	\label{eq:pseudo}
\end{align}


\subsection{Best practices for tables}
\label{sec:tables}

% \begin{table}
	% \renewcommand{\arraystretch}{1.3} % spacing between rows
	% \centering
	\begin{tabular}{ @{} llll @{} } \toprule % @{} removes space
		Element 
		& Core MF 
		& Mantle MF 
		& $C_\text{cap}^N (\text{s}^{-1})$ 
		\\ \hline
		Iron 
		& 0.855 
		& 0.0626 
		& $9.43\times 10^{7}$ 
		\\
		Nickel 
		& 0.052 
		& 0.00196 
		& $7.10\times 10^{6}$ 
		\\
		Silicon 
		& 0.06 
		& 0.210 
		& $2.24\times 10^{6}$ 
		\\
		Magnesium 
		& 0 
		& 0.228 
		& $1.05\times 10^{6}$ 
		\\ \bottomrule
	\end{tabular}
	% \caption{
		% Mass fractions of the Earth's core and mantle.
		% \label{table:elements}
% 	}
% \end{table}




\section{Labels and cleveref}
\label{sec:labels:and:cleveref}


\subsection{\texorpdfstring{\texttt{cleveref}}{cleveref}}
\label{sec:cleveref}

\texttt{cleveref} is a handy package when referring to ranges of equations. 

The pseudoscalar rule is:
\begin{itemize}
	\item Using \texttt{amsmath.sty}'s \texttt{eqref}: \eqref{eq:pseudo}
	\item Using \texttt{cleverefs}'s \texttt{cref}: \cref{eq:pseudo}
\end{itemize}

The equations above are
\begin{itemize}
	\item Using \texttt{amsmath.sty}'s \texttt{eqref}: \eqref{eq:vector} -- \eqref{eq:pseudo}
	\item Using \texttt{cleverefs}'s \texttt{crefrange}: \crefrange{eq:vector}{eq:pseudo}
\end{itemize}

The equations above are (glomped together)
\begin{itemize}
	\item Using \texttt{amsmath.sty}'s \texttt{eqref}: \eqref{eq:vector}, \eqref{eq:axial}, \eqref{eq:scalar}, \eqref{eq:pseudo}
	\item Using \texttt{cleverefs}'s \texttt{cref}: \cref{eq:vector,eq:pseudo,eq:axial,eq:scalar}
\end{itemize}

\texttt{cleveref} automatically identifies the type of object it is referring to. Thus you can use \textbackslash\texttt{cref} to refer to any label, for example \cref{foot:centering}. You can use \textbackslash\texttt{Cref} to have a capitalized the cross reference name. For example: the sections above are \Cref{sec:macros,sec:cleveref,sec:figs,sec:tables}.



\subsection{Sub-equations}

One can also wrap a \texttt{align} environment with a \texttt{subequations} environment. The \texttt{subequations} environment can be given a label. For example,
\begin{subequations}\label{eq:subequations}
\begin{align}
	a &= \pi 
	\label{eq:subequation:1}
	\\
	b &= \e^{i\pi} 
	\ .
	\label{eq:subequation:2}
\end{align}
\end{subequations}
where we can now refer to the pair of equations \eqref{eq:subequations} or simply one of the equations, \eqref{eq:subequation:2}. This also works in \texttt{cleveref}: \cref{eq:subequations} and \cref{eq:subequation:2}.


\subsection{Referring to Equations}

One style suggestion is to use parentheses to refer to an equation with no additional modifiers \emph{except} at the beginning of a sentence.\footnote{\url{https://academia.stackexchange.com/a/21793}} For example: ``The second term in (3)...'' and ``Equation~(3) has two terms...''



\section{Macros}
\label{sec:macros}


\subsection{Modest capitalization}

Small caps are useful when your text contains acronyms and you do not want them to visually imply emphasis. In other words, we can use them as `not shouting' capitalization. We define a macro \texttt{acro} for this purpose. The default is for \texttt{acro} to be a wrapper for \texttt{small}. Here is an example:
\begin{itemize}
	\item \acro{AdS} in \acro{5D} at the \acro{LHC}.
	\item AdS in 5D at the LHC. 
	\item A third list item to test list spacing.
\end{itemize}


\subsection{Comments for Collaboration}

There are many ways to add notes when collaborating on a document. I like in-line notes with an author name and a color.  \flip{This is an example of a comment.} It is also useful to have a macro for highlighting new text and for proposing the removal of old text.

\new{
I fixed the equation:
\begin{align}
	a=b^2 
	\label{eq:samename}
\end{align}
It has label \texttt{eq:samename}. 
}

\remove{
	% It is good practice to indent the `to-be deleted' text
	Here is an equation:
	\begin{align}
	a=b 
	\label{eq:samename}
	\end{align}
	It has label \texttt{eq:samename}.
}

This is essentially a manual version of the \texttt{latexdiff} command. This command can be notoriously fussy around math environments. I personally advocate for using \texttt{git}-related tools to quickly identifying where a version was edited and then using tags to identify edits that need to be highlighted for further discussion. One nice thing about the \texttt{remove} tag above is that it also strips any \texttt{label}s so that there are no `multiply defined label' arguments and one can uniquely refer to a single equation, \eqref{eq:samename}.

\begin{flipcomment}
This is an extended comment that shows up as a text box. I might use this to make some ponderous point about why I think my version of a draft paragraph is more appropriate than yours.
\end{flipcomment}


\subsection{Writing Macros}

The \textbackslash\texttt{xspace} command is useful at the end of a macro. It stands for: ``insert a space if and only if there is supposed to be a space.'' Consider the following examples:
\begin{itemize}
	\item Without \texttt{xspace}: \LaTeX typesetting...
	\item With \texttt{xspace}: \LaTeX\xspace typesetting...
	\item Without \texttt{xspace}: Typeset with \LaTeX.
	\item With \texttt{xspace}: Typeset with \LaTeX\xspace.
\end{itemize}


\section{Mathematics and Physics}

\subsection{Environments}

Use the \texttt{align} environment instead of \texttt{eqnarray}, see the TeXblog discussion\footnote{\url{https://texblog.net/latex-archive/maths/eqnarray-align-environment/}}. The double dollar sign notation for \texttt{displaymath} is depreciated.\footnote{\url{https://tex.stackexchange.com/questions/503/why-is-preferable-to}} The suggested alternative is \textbackslash\texttt{[} and \textbackslash\texttt{]}, though this is rather annoying. As a default I always use \texttt{align}.


\subsection{Text and Math: super- and subscripts}

Use the \texttt{text} command to insert text into math environments. For subscripts use \texttt{textnormal}. 
% 
We use the \texttt{textnormal} command for super- or subscripts rather than the \texttt{text} command because this automatically uses the correct size. Compare, for example: $G_\textnormal{D},\, G_\text{D}$.

One place this shows up is if you have a subscript that is not a mathematical variable. For example, $x_a$ makes sense for the value of $x$ at point $a$, but $x_\text{b}$ should be used if the `b' is shorthand for boundary. Similarly, $E_\text{max}$ for the maximum energy, rather than $E_{max}$.


\subsection{Units and spacing}

Use a tie (tilde) to enforce a non-breaking short space between a number and its units: $0.5~\text{MeV}$. Units should not be italicized.


\subsection{Upright characters}

These come from the \acro{ISO 80000} standards for typesetting mathematics and physics.\footnote{See discussion in \url{https://tex.stackexchange.com/questions/14821/whats-the-proper-way-to-typeset-a-differential-operator}} It is not obvious to me that these are applicable to the typographical culture of physics, but the most important thing is to be consistent.
\begin{itemize}
	\item Units are always upright, \textmu m is a micrometer. You can use various unit packages to do this automatically. 
	\item The base of the natural logarithm is upright $\e^{i\pi}$ versus $e^{i\pi}$.
	\item The differential is an operator so it should be upright. I have macros for this: $dx$ vs.~$\D{x}$ and $\dbar p$ vs.~$\Dbar{p}$. The \texttt{physics} package has macros for this. 
\end{itemize}
There is a historical discussion on \texttt{hsm.stackexchange}\footnote{\url{https://hsm.stackexchange.com/questions/6727/fracdydx-versus-frac-mathrm-dy-mathrm-dx}}.



\subsection{Miscellaneous}


\begin{itemize}
	\item Use $\mid$ instead of pipe for conditions: $p(x\mid y)$ versus $p(x|y)$.
	% 
	\item Transpose: The \acro{ISO}~80000 standard has suggestions. $A^T$ vs $A^\top$ vs $A^{\trans}$. I personally prefer $A^\text{T}$.
	% 
	\item \textbf{Textual subscripts}: sometimes you have a subscript that is not an index, but shorthand for something textual. For example, the Green's function with Dirichlet boundary conditions is $G_\textnormal{D}$. The subscript is upright, not italicized, $G_D$. Use the \texttt{textnormal} command rather than the \texttt{text} command since this will automatically use the correct size. Comparison: $G_\textnormal{D},\, G_\text{D}$.
	% 
	\item Arrows with text under them: \texttt{xrightarrow} in the \texttt{amsmath} package. The square bracketed argument is under, the curly bracketed argument is over. Example: $\xrightarrow[\textnormal{low}]{\textnormal{hi}}$.
	% 
\end{itemize}


\section{Space}

\subsection{Kerning: spacing between characters}
Math operators have a natural spacing before and after depending on the context. In the following example, spaces indicate that the coefficient $a$ multiplies the logarithm of $b$:
\begin{align}
	a\log b && a\log(bc)
\end{align}
The space on either side of $\log$ indicate that $\log$ is a mathematical operator. The second example still has the space between $a$ and $\log$, but has no space between $\log$ and $(bc)$ because the parenthesis belongs to the mathematical function.\footnote{Example from \url{https://tex.stackexchange.com/a/140647}}
% 
You can use \textbackslash\texttt{DeclareMathOperator} for functions that are not built in.


\subsection{Manual Spacing}

\LaTeXx has commands for manually inserting spacing: thin\,space, medium\:space, thick\;space, thin\!negative\!space.


\subsection{Ties create non-breaking spaces}

\LaTeXx interprets periods as full stops (end of a sentence). It places extra space after the full stop. Use a \textbf{tie} (tilde) when a period is not a full stop: Mr.~Roboto versus Mr. Roboto. 
% 
You can also use ties this whenever you want to prevent a line break between words. \LaTeXx interprets the tied words as a single word.
% 
Ties are also standard for citations$\sim$\textbackslash\texttt{cite}\{\texttt{citation}\}.



\section{Some Best Practices}

\subsection{\texorpdfstring{\LaTeXx in a title}{LaTeX in a title}}

When using \LaTeXx code in a section title, use the \texttt{texorpdfstring} command to define an \acro{UTF-8} string that the pdf can use for bookmarks. If you do not do this, there are annoying compilation warnings.


\subsection{Ranges}

Hyphens, en dashes, em dashes, and minus signs in math mode are all grammatically different.
\begin{itemize}
	\item En dashes replace hyphens in a compound adjective where one of the elements is a two-word compound: `post--Cold War era.'\footnote{From \url{https://www.merriam-webster.com/words-at-play/em-dash-en-dash-how-to-use}}
	\item En dashes are used for combinations of two names in place of the wordd `and'. Randall--Sundrum model.
	\item For compound names of a single person, use a hyphen: Levi-Civita.
\end{itemize}



% \input{examples_teaching}
% \input{examples_listings}
% %!TEX root = paper.tex
%% Update the above with the appropriate root

\section{\texorpdfstring{\LaTeX\ Style}{LaTeX Style}}

There is not a definitive \LaTeX\ style guide analogous to \acro{PEP-8}. However, I do have my own set of preferences. Here style refers to how the \LaTeX\ source files are written. Two \texttt{tex} files may produce identical \texttt{pdf} outputs but be stylistically different. A well styled document is that is as easy as possible to parse and edit as a human being.


\subsection{Idiosyncracies}

While some coding style guides require a fixed width for the document. This gives meaning to a line of code and makes the resulting source more readable by avoiding unintentional text wrapping. \LaTeX is a bit different in that it is a typesetting language that is meant to handle paragraphs of text. Because modern editors naturally have text wrapping options, I do not feel strongly about enforcing a document-wide character width limit. Paragraphs of text should be allowed to wrap if that makes sense. However, mathematics environments should strive to make use of white space in service to readability.


\subsection{Spacing and Indents}

White space helps distinguish the document structure. 

\begin{enumerate}
	\item Sections should have three empty lines between each other.
	\item Sub-sections should have two empty lines between each other.
	\item All other units of paragraphs should have one empty line space between them.
\end{enumerate}
One may use with commented out empty lines to separate sentences from one another. 
% 
	This produces no paragraph break between the sentences in the output, but can help separate different ideas within a paragraph.
% 
	Similarly, one can combine this with indents to help visually organize the logical flow of a paragraph.


\subsection{Mathematics}

Use comments and white space to separate mathematics environments from plain text.
% 
The contents of a mathematics environment should use ample white space to separate each mathematical object as if these were words in a sentence.
% 
\begin{lstlisting}[style=latexstyle]
\begin{align}
	S_{\textnormal{fix}}^{\textnormal{Bulk}}
	& =
	\frac{-1}{g^2} 
	\int d^{d+1} x  
	\left( \frac{R}{z} \right)^{\!d-3}
	\frac{1}{2\xi}
	\left[
	    \partial_\mu A^\mu
	    -
	    \xi\left( 
	        z^{d-3} 
	        \partial_z \left( \frac{A_z}{z^{d-3}} \right)
	        -
	        \left(\frac{R}{z}\right)^{\!2} g^2 v(z)\, \pi
	    \right)
	\right]^2 \ ,
\label{eq:SGFBulk}
\end{align}
\end{lstlisting}
% 
This produces the following:
\begin{align}
	S_{\textnormal{fix}}^{\textnormal{Bulk}}
	& =
	\frac{-1}{g^2} 
	\int d^{d+1} x  
	\left( \frac{R}{z} \right)^{\!d-3}\frac{1}{2\xi}
	\left[
	    \partial_\mu A^\mu
	    -
	    \xi\left( 
	        z^{d-3} \partial_z \left(\frac{A_z}{z^{d-3}}\right)
	        -
	        \left(\frac{R}{z}\right)^{\!2} g^2 v(z)\, \pi
	    \right)
	\right]^2 \ ,
% \label{eq:SGFBulk}
\end{align}
For long expressions, each line should be a well-defined `unit' of the mathematical expression. When there are multiple `words' in an expression, add white space to delimit them. 

For example, simple fractions do not need any white space, while more complicated fractions should provide some help:
% 
\begin{lstlisting}[style=latexstyle]
\frac{ 
	b_\mathcal{O}
}{
	\Lambda^{\Delta_\mathcal{O} - \frac{d}{2} - 1 }
} 
\end{lstlisting}
% 
This may seem like overkill, but in an expression with multiple fractions it is helpful to be able to quickly visually parse each piece of the expression.



% %!TEX root = paper.tex
%% Update the above with the appropriate root

\section{References}

\subsection{\texorpdfstring{\LaTeXx Style}{LaTeX Style}}

\begin{itemize}
    \item Didier Verna, ``Towards \LaTeXx coding standards\footnote{\url{https://tug.org/TUGboat/tb32-3/tb102verna.pdf}; video: \url{http://zeeba.tv/toward-latex-coding-standards/}}.'' 
    \item See also Philippe Beliveau's summary of Verna's piece.\footnote{\url{https://medium.com/@pbeliveau/latex-coding-standards-f82743b7866b}}
    \item Evan Chen, ``Evan's \LaTeXx Style Guide.\footnote{\url{https://web.evanchen.cc/latex-style-guide.html}}''
\end{itemize}


\subsection{Style Guides}

Most publications have a style guide, analogous to the well-known \acro{APA} style in social sciences. Checking for strict adherence to those guides have historically been the role of copy editors, though the number of published papers continues to grow with no obvious increase in resources for copy editing. 

\begin{itemize}
    \item There is an \acro{ISO} standard for typesetting mathematics and physics. As of 2023 it is \acro{ISO80000}.
    \item Strunk \& White, \emph{The Elements of Style}
    \item The \emph{Review of Modern Physics} style guide. 
\end{itemize}

\subsection{Typography References}

The standard typography reference is Robert Bringhurst's \emph{The Elements of Typographic Style}. 
The following references focus specifically on typography and \LaTeXx.
\begin{itemize}
    \item Consistent typography on \acro{TeX.SE}\footnote{\url{https://tex.stackexchange.com/questions/29840/consistent-typography}}
    \item List of best practices references on \acro{TeX.SE}\footnote{\url{https://tex.stackexchange.com/questions/577/best-practices-references}}, including the list of obsolete packages and commands in \LaTeXx~2e\footnote{\url{https://www.ctan.org/tex-archive/info/l2tabu/english/}}
    \item ``The Art of \LaTeXx,'' a list of guidelines from Fan Pu Zeng\footnote{\url{https://fanpu.io/blog/2023/latex-tips/}}
    \item Showcase of beautiful typography in \LaTeXx\footnote{\url{https://tex.stackexchange.com/q/1319/8094}}
\end{itemize}



%% CHAPTER SUBAPPENDIX %% if using report class
% \begin{subappendices}
% \section{Parenthetical}
% \end{subappendices}


\chapter{More examples}

% \newtheorem{theorem}{Theorem}[section]
% \newtheorem{exercise}{Exercise}[section]
% \newtheorem{example}{Example}[section]
% \newtheorem{bigidea}{Key Idea}[section]
%     \newcommand{\bigidearef}{Key~Idea}
%     \newcommand{\bigidearefs}{Key~Ideas}

\section{Some common environments}

\begin{theorem}[Euler's Identity]
\label{thm:euler:identity}
    Euler's identity is
    \begin{align}
        \e^{i\pi} = -1 \ .
    \end{align}
    Note that we use the macro \verb!\e! for an upright $\e$ rather than an italicized $e$.
\end{theorem}

\begin{exercise}
\label{ex:derive:euler:identity}
    Derive Euler's identity, Thm.~\ref{thm:euler:identity}.
\end{exercise}

\noindent Good students do the exercises, like Exercise~\ref{ex:derive:euler:identity}. Good instructors provide lots of examples, like Example~\ref{eg:easy:example}.

\begin{example}
\label{eg:easy:example}
    Consider the geometric series
    \begin{align}
        S = \sum_{n=0} a^n \ .
    \end{align}
    We can find a closed form expression for $S$ using
    \begin{align}
        S - aS &= 1\\
        S &= \frac{1}{1-a} \ .
    \end{align}
\end{example}

\section{Some specialized environments}

\begin{bigidea}[Principle of Easy Examples]
\label{idea:easy:examples}
Don't you hate it when the examples are so much easier than the exercises?
\end{bigidea}

\noindent The \bigidearef{}~\ref{idea:easy:examples} is one we may want to refer to later.

\chapter{New chapter}

Test cite\autocite{Feng:2016ijc}
\lipsum[1]
And cite again\autocite{Feng:2016ijc}.
\sidenote{Another} \lipsum[1]\footnote{footnote}

\section{And more stuff}
\lipsum[3]
And cite again\autocite{Feng:2016ijc}.


\chapter{Anther chapter}
\lipsum[3]\autocite{Feng:2016ijc}

\section*{Acknowledgments}

\acro{PT}\ thanks 
\emph{your name here}
for useful comments and discussions. 
%
\acro{PT} thanks 
    the Aspen Center for Physics (\acro{NSF} grant \acro{\#1066293})
    % and the Kavli Institute for Theoretical Physics (\acro{NSF} grant \acro{PHY-1748958})`'
    for 
    its 
    % their
    hospitality during a period where part of this work was completed. 
%
% \acro{PT} is supported by the \acro{DOE} grant \acro{DE-SC}/0008541.
\acro{PT} is supported by a \acro{NSF CAREER} award (\#2045333).

%% Appendices
% \appendix


%% Bibliography
%% USING BIBLATEX, SKIP BIBTEX
%% Use inspireHEP bibtex entries when possible
% \bibliographystyle{utcaps} 	% arXiv hyperlinks, preserves caps in title
% \bibliography{bib title without .bib}


\end{document}