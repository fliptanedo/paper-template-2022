%!TEX root = paper.tex
%% FLIP’S MACROS 
%% USES: FlipPreamble.tex

%% FOR `NOT SHOUTING' CAPS (e.g. acronyms)
%% ---------------------------------------
\usepackage{scalefnt} 
% \newcommand\acro[1]{{\footnotesize #1}}           % acronyms in footnote size
\newcommand\acro[1]{{\small{#1}}} 


%% COMMANDS FOR TEMPORARY COMMENTS
%% -------------------------------
\newcommand{\comment}[2]{\textcolor{red}{[\textbf{#1} #2]}}
\newcommand{\flip}[1]{{
  \color{green!50!black}
  \footnotesize
  [\textbf{\textsf{Flip}}: \textsf{#1}]
  }}

\newcommand{\new}[1]{{ 
    \color{green!50!black}\footnotesize
    [\textbf{\textsf{New}}: {#1}]}}

%% FRAMES (mdframed package)
\global\mdfdefinestyle{flipbox}{%
  linecolor=green!50!black,linewidth=1pt,%
  leftmargin=0cm, rightmargin=0cm
  } 


\newenvironment{flipcomment}
  {
    \begin{mdframed}[style=flipbox] 
    \color{green!50!black}
    \sffamily
    \footnotesize
    \textbf{\textsf{Flip}}:
  }{
    \end{mdframed}
  }

%% Because mdframed causes a bunch of warnings
%% https://tex.stackexchange.com/questions/64331/disable-warning-from-mdframed 
\usepackage{silence}
\WarningFilter{mdframed}{You got a bad break}
\WarningFilter{mdframed}{correct box splittet fails}



%% REMOVE Environment
%% https://tex.stackexchange.com/a/488582/8094
%% Creates a nolabel environment that strips all labels
%% This is useful to avoid multiple label definitions
%% When marking old versions for deletion
\usepackage{xparse}
\ExplSyntaxOn
\NewDocumentEnvironment{nolabel}{}{
  \cs_set_eq:NN \label \use_none:n
  \cs_set_eq:cN { ltx@label} \use_none:n
}{}
\ExplSyntaxOff 

\newcommand{\remove}[1]{{
  \begin{nolabel} 
    \color{red!50!black}\footnotesize
    [\textbf{\textsf{Removed}}: {#1}]
    \end{nolabel}
    }}



%% COMMON PHYSICS MACROS
%% ---------------------
\renewcommand{\tilde}{\widetilde}                 % tilde over characters
\renewcommand{\text}{\textnormal}	                % text in equations 
\renewcommand{\vec}[1]{\mathbf{#1}}               % vectors are boldface

%% Differential and differential/2pi
% \newcommand{\dbar}{d\mkern-6mu\mathchar'26}     % for d/2pi
\newcommand{\dbar}{d\mkern-6mu\mathchar'26\hspace{-.1em}}    

%% Best practice: Roman differential
\newcommand{\D}[1]{\ensuremath{\operatorname{d}\!{#1}}}
\newcommand{\DD}[2]{\ensuremath{\operatorname{d}^{#1}\!{#2}}}
\newcommand{\Dbar}[1]{\operatorname{d}\mkern-10mu\mathchar'26\mkern-2mu{#1}} 

\newcommand{\ket}[1]{\left|#1\right\rangle}       % <#1|
\newcommand{\bra}[1]{\left\langle#1\right|}       % |#1>

%% Best practice: base of natural log is Roman
\newcommand{\e}{\operatorname{e}}  

%% Best practice: imaginary number is Roman too!?
\newcommand{\I}{\operatorname{i}\mkern-2mu}  


%% Matrices
%% --------
\newcommand{\aij}[2]{^{#1}_{\phantom{#1}#2}}
\newcommand{\mat}[3]{#1\aij{#2}{#3}}
\newcommand{\pp}{\phantom{+}}                     % phantom + for spacing
\newcommand{\inv}{^{-1}}
\newcommand{\one}{\mathbbm{1}}
\newcommand{\Tr}{\text{Tr}\,}
\newcommand{\RR}{\mathbbm{R}}
\newcommand{\CC}{\mathbbm{C}}
\newcommand*{\trans}{\mkern-1.5mu\mathsf{T}}      % transpose


% Make my life easer
% ------------------
\newcommand{\la}{\langle}
\newcommand{\ra}{\rangle}
\newcommand*{\smallslot}{\,\underline{\makebox[0.80em]{\ensuremath{}}}\,}


\usepackage{scalerel} % https://tex.stackexchange.com/a/523873/8094
\newcommand*{\paral}{{\stretchrel*{\parallel}{\perp}}}


\usepackage{pifont}
  \newcommand{\cmark}{\ding{51}}%
  \newcommand{\xmark}{\ding{55}}%