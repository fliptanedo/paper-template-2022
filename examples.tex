%!TEX root = paper.tex
%% Update the above with the appropriate root

\section{Macros}
\label{sec:macros}

\flip{This is a comment. Let's test out the `not shouting' caps:}
\begin{itemize}
	\item \acro{AdS} in \acro{5D} at the \acro{LHC}.
	\item AdS in 5D at the LHC. 
	\item A third list item to test list spacing.
\end{itemize}


\section{Figures in Equation Environments}
\label{sec:figs}

\begin{align}
	\vcenter{
		\hbox{\includegraphics[width=.1\textwidth]{{example-image-a}}}
		}
	&=
	i g \gamma^\mu \ . 
	\label{eq:vector}
	\\
	\vcenter{
		\hbox{\includegraphics[width=.1\textwidth]{{example-image-a}}}
		}
	&=
	g \gamma^\mu\gamma^5 \ . 
	\label{eq:axial}
	\\
	\vcenter{
		\hbox{\includegraphics[width=.1\textwidth]{{example-image-a}}}
		}
	&=
	ig  \ . 
	\label{eq:scalar}
	\\
	\vcenter{
		\hbox{\includegraphics[width=.1\textwidth]{{example-image-a}}}
		}
	&=
	g \gamma^5 \ . 
	\label{eq:pseudo}
\end{align}

\section{Best practices for tables}
\label{sec:tables}

% \begin{table}
	% \renewcommand{\arraystretch}{1.3} % spacing between rows
	% \centering
	\begin{tabular}{ @{} llll @{} } \toprule % @{} removes space
		Element & Core MF & Mantle MF & $C_\text{cap}^N (\text{s}^{-1})$ 
		\\ \hline
		Iron & 0.855 & 0.0626 & $9.43\times 10^{7}$ 
		\\
		Nickel & 0.052 & 0.00196 & $7.10\times 10^{6}$ 
		\\
		Silicon & 0.06 & 0.210 & $2.24\times 10^{6}$ 
		\\
		Magnesium & 0 & 0.228 & $1.05\times 10^{6}$ 
		\\ \bottomrule
	\end{tabular}
	% \caption{
		% Mass fractions of the Earth's core and mantle.
		% \label{table:elements}
% 	}
% \end{table}

\section{cleveref}
\label{sec:cleveref}

\texttt{cleveref} is a handy package when referring to ranges of equations. 

The pseudoscalar rule is:
\begin{itemize}
	\item Using \texttt{amsmath.sty}'s \texttt{eqref}: \eqref{eq:pseudo}
	\item Using \texttt{cleverefs}'s \texttt{cref}: \cref{eq:pseudo}
\end{itemize}

The equations above are
\begin{itemize}
	\item Using \texttt{amsmath.sty}'s \texttt{eqref}: \eqref{eq:vector} -- \eqref{eq:pseudo}
	\item Using \texttt{cleverefs}'s \texttt{crefrange}: \crefrange{eq:vector}{eq:pseudo}
\end{itemize}

The equations above are (glomped together)
\begin{itemize}
	\item Using \texttt{amsmath.sty}'s \texttt{eqref}: \eqref{eq:vector}, \eqref{eq:axial}, \eqref{eq:scalar}, \eqref{eq:pseudo}
	\item Using \texttt{cleverefs}'s \texttt{cref}: \cref{eq:vector,eq:pseudo,eq:axial,eq:scalar}
\end{itemize}

The sections above are \cref{sec:macros,sec:cleveref,sec:figs,sec:tables}.

\section{Other Best Practices}


\subsection{Kerning: spacing between characters}
Math operators have a natural spacing before and after depending on the context. In the following example, spaces indicate that the coefficient $a$ multiplies the logarithm of $b$:
\begin{align}
	a\log b && a\log(bc)
\end{align}
The space on either side of $\log$ indicate that $\log$ is a mathematical operator. The second example still has the space between $a$ and $\log$, but has no space between $\log$ and $(bc)$ because the parenthesis belongs to the mathematical function.\footnote{Example from \url{https://tex.stackexchange.com/a/140647}}

You can use \textbackslash\texttt{operatorname} and \textbackslash\texttt{DeclareMathOperator} for functions that are not built in.

\subsection{Upright characters}
These come from the \acro{ISO 80000} standards for typesetting mathematics and physics.\footnote{See discussion in \url{https://tex.stackexchange.com/questions/14821/whats-the-proper-way-to-typeset-a-differential-operator}} It is not obvious to me that these are applicable to the typographical culture of physics, but the most important thing is to be consistent.
\begin{itemize}
	\item Units are always upright, \textmu m is a micrometer. You can use various unit packages to do this automatically. 
	\item The base of the natural logarithm is upright $\mathrm{e}^{i\pi}$ versus $e^{i\pi}$.
	\item The differential is an operator so it should be upright. I have macros for this: $dx$ vs.~$\D{x}$ and $\dbar p$ vs.~$\Dbar{p}$. The \texttt{physics} package has macros for this. 
\end{itemize}
There is a historical discussion on hsm.stackexchange\footnote{\url{https://hsm.stackexchange.com/questions/6727/fracdydx-versus-frac-mathrm-dy-mathrm-dx}}.


\subsection{Ranges}

Hyphens, en dashes, em dashes, and minus signs in math mode are all grammatically different.
\begin{itemize}
	\item En dashes replace hyphens in a compound adjective where one of the elements is a two-word compound: `post--Cold War era.'\footnote{From \url{https://www.merriam-webster.com/words-at-play/em-dash-en-dash-how-to-use}}
	\item En dashes are used for combinations of two names in place of the wordd `and'. Randall--Sundrum model.
	\item For compound names of a single person, use a hyphen: Levi-Civita.
\end{itemize}

\subsection{Inserting code}

Uncomment the following as well as the lines in \texttt{paper.tex} that loads the \texttt{listings} package.
% %!TEX root = paper.tex
%% Update the above with the appropriate root

\section{Code example}

These are examples of the \texttt{listings} package for typesetting code. See Overleaf\footnote{\url{https://www.overleaf.com/learn/latex/Code_listing}} and tex.stackexchange\footnote{\url{https://tex.stackexchange.com/a/350242}} for discussions.

\subsection{Basic Example}
\begin{pyin}
print("Hello world")
\end{pyin}

\begin{pyprint}
Hello world
\end{pyprint}

\subsection{Outputs and returned values}
And here we also have a return value in the last line of the input cell.
\begin{pyin}[labelOfTheSecondInput]
def twicify(arg):
    print("Received:", arg, "- Will double now...")
    return 2 * arg
twicify(1)
\end{pyin}

\begin{pyprint}
Received: 1 - Will double now...
\end{pyprint}

\begin{pyout}
2
\end{pyout}

\subsection{Referencing input}
You can also reference the labeled input \ref{labelOfTheSecondInput}, from above.
\begin{pyin}
"and the counter will automatically do the right thing :)"
\end{pyin}
\begin{pyout}
'and the counter will automatically do the right thing :)'
\end{pyout}




\subsection{Miscellaneous}

\begin{itemize}
	\item \textbf{align environment}: Use the \texttt{align} environment instead of \texttt{eqnarray}. See TeXblog discussion\footnote{\url{https://texblog.net/latex-archive/maths/eqnarray-align-environment/}}
	\item \textbf{centering environment}: Use \textbackslash\texttt{centering} rather than the \texttt{center} environment in figure environments to avoid adding extra vertical space.\footnote{\url{https://tex.stackexchange.com/questions/23650/when-should-we-use-begincenter-instead-of-centering}}
	\item \textbf{Roman Subscripts}: use Roman style (e.g.~\textbackslash \texttt{text}\{$\cdots$\}) when the subscript is not a mathematical character. For example, $x_a$ makes sense for the value of $x$ at point $a$, but $x_\text{b}$ should be used if the `b' is shorthand for boundary. Similarly, $E_\text{max}$ for the maximum energy, rather than $E_{max}$.
	\item Spacing: thin\,space, medium\:space, thick\;space, thin\!negative\!space.
	\item \textbf{Ties} (tildes): \LaTeX interprets periods as full stops (end of a sentence). It places extra space after the full stop. Use a tie when a period is not a full stop: Mr.~Roboto versus Mr. Roboto. 
	\item You can also use ties this whenever you want to prevent a line break between words. \LaTeX interprets the tied words as a single word.
	\item The double dollar sign notation for \texttt{displaymath} is depreciated.\footnote{\url{https://tex.stackexchange.com/questions/503/why-is-preferable-to}}
	\item Use $\mid$ instead of pipe for conditions: $p(x\mid y)$ versus $p(x|y)$.
	\item Transpose: The \acro{ISO}~80000 standard has suggestions. $A^T$ vs $A^\top$ vs $A\trans$
\end{itemize}

More information available here:
\begin{itemize}
	\item Consistent typography on tex.stackexchange\footnote{\url{https://tex.stackexchange.com/questions/29840/consistent-typography}}
	\item List of best practices references on tex.stackexchange\footnote{\url{https://tex.stackexchange.com/questions/577/best-practices-references}}, including the list of obsolete packages and commands in 2e\footnote{\url{https://www.ctan.org/tex-archive/info/l2tabu/english/}}
	\item There is an \acro{ISO} standard for typesetting mathematics and physics. As of 2023 it is \acro{ISO\~80000}.
	\item ``The Art of \LaTeX,'' a list of guidelines from Fan Pu Zeng\footnote{\url{https://fanpu.io/blog/2023/latex-tips/}}
	\item Showcase of beautiful typography in \LaTeX\footnote{\url{https://tex.stackexchange.com/questions/1319/showcase-of-beautiful-typography-done-in-tex-friends}}
\end{itemize}
